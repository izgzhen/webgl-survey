\section{Compiled Information from
MDN}\label{compiled-information-from-mdn}

Source: \href{https://developer.mozilla.org/en-US/docs/Web/API}{API},
\href{https://developer.mozilla.org/en-US/docs/Web/API/WebGL_API/WebGL_best_practices}{Best
Practices}

\subsection{Source of bugs}\label{source-of-bugs}

\subsubsection{canvas element}\label{canvas-element}

To get the rendering context, i.e., the canvas element, we usually use
\texttt{getElementById} or similar way to fetch the DOM object. The
caveat is, the \texttt{id} is specified in HTML tag, such as

\begin{verbatim}
<canvas id="glcanvas" width="640" height="480">
\end{verbatim}

And you must use the exact name in your JavaScript code. \textbf{It is
possible that you will type it wrongly or refer to another unrelated
canvas}.

\subsubsection{\texorpdfstring{\texttt{gl}
Object}{gl Object}}\label{gl-object}

\begin{verbatim}
function initWebGL(canvas) {
  gl = null;

  try {
    // Try to grab the standard context. If it fails, fallback to experimental.
    gl = canvas.getContext("webgl") || canvas.getContext("experimental-webgl");
  }
  catch(e) {}

  // If we don't have a GL context, give up now
  if (!gl) {
    alert("Unable to initialize WebGL. Your browser may not support it.");
    gl = null;
  }

  return gl;
}
\end{verbatim}

First, we might not have a functional \texttt{gl} (not \texttt{null})
always, which depends on the browser support. And second, in the
\texttt{getContext}, we are also forced to consider two standards.

\subsubsection{Shader's source}\label{shaders-source}

The shader code is essentially a string. So as a result, you might
either define it in JavaScript, or store it as a HTML element, or even
request it as a resource dynamically from other address.

\begin{verbatim}
if (!gl.getProgramParameter(shaderProgram, gl.LINK_STATUS)) {
  alert("Unable to initialize the shader program.");
}
\end{verbatim}

The above snippet checks if the \texttt{gl.linkProgram} calling succeeds
by checking if return value is \texttt{null}. It is apparently something
easy to forget.

\subsubsection{Attribute names}\label{attribute-names}

The shader code will use some variable to communicate with JavaScript.
For example

\begin{verbatim}
<script id="shader-vs" type="x-shader/x-vertex">
  attribute vec3 aVertexPosition;

  uniform mat4 uMVMatrix;
  uniform mat4 uPMatrix;

  void main(void) {
    gl_Position = uPMatrix * uMVMatrix * vec4(aVertexPosition, 1.0);
  }
</script>
\end{verbatim}

On the JavaScript side, we have to

\begin{verbatim}
// During Shader initialization
vertexPositionAttribute = gl.getAttribLocation(shaderProgram, "aVertexPosition");
gl.enableVertexAttribArray(vertexPositionAttribute);

// During scene rendering
var pUniform = gl.getUniformLocation(shaderProgram, "uPMatrix");
gl.uniformMatrix4fv(pUniform, false, new Float32Array(perspectiveMatrix.flatten()));

var mvUniform = gl.getUniformLocation(shaderProgram, "uMVMatrix");
gl.uniformMatrix4fv(mvUniform, false, new Float32Array(mvMatrix.flatten()));

gl.bindBuffer(gl.ARRAY_BUFFER, squareVerticesBuffer);
gl.vertexAttribPointer(vertexPositionAttribute, 3, gl.FLOAT, false, 0, 0);
\end{verbatim}

So, we have to match the names (i.e. \texttt{aVertexPosition},
\texttt{uPMatrix} and \texttt{uMVMatrix} in the above example) in two
language domains, both in syntax and semantics.

\subsubsection{Array}\label{array}

This is how to fill data into GL buffer:
\texttt{gl.bufferData(gl.ELEMENT\_ARRAY\_BUFFER,\ new\ Uint16Array(cubeVertexIndices),\ gl.STATIC\_DRAW);}

The \texttt{Uint16Array} is a raw, platform-dependent way of storing an
array of data. Similarly, we also have \texttt{Array},
\texttt{Int8Array}, \texttt{Float32Array} \ldots{}

Interestingly, let compare it with OpenGL ES interface:
\texttt{void\ BufferData(\ enum\ target,\ sizeiptr\ size,\ const\ void\ *data,\ enum\ usage\ );}.

You can see that, the \texttt{Uint16} array could be translated into a
raw array and a element size indicator.

\subsection{Things to avoid}\label{things-to-avoid}

\begin{itemize}
\tightlist
\item
  You should never use \texttt{\#ifdef\ GL\_ES} in your WebGL shaders;
  although some early examples used this, it's not necessary, since this
  condition is always true in WebGL shaders.
\item
  Using \texttt{high} precision in fragment shaders will prevent your
  content from working on some older mobile hardware. You can use
  \texttt{medium} instead, but be aware that this often results in
  corrupted rendering due to lack of precision on most mobile devices,
  and the corruption is not going to be visible on a typical desktop
  computer. In general, only using \texttt{high} in both vertex and
  fragment shaders is safer unless shaders are thoroughly tested on a
  variety of platforms. Starting in Firefox 11, the WebGL
  \texttt{getShaderPrecisionFormat()} function is implemented, allowing
  you to check if \texttt{high} precision is supported, and more
  generally letting you query the actual precision of all supported
  precision qualifiers.
\item
  Anything that requires syncing the CPU and GPU sides is potentially
  very slow, so if possible you should try to avoid doing that in your
  main rendering loops. This includes the following WebGL calls:
  \texttt{getError()}, \texttt{readPixels()}, and \texttt{finish()}.
  WebGL getter calls such as \texttt{getParameter()} and
  \texttt{getUniformLocation()} should be considered slow too, so try to
  cache their results in a JavaScript variable.
\item
  Simpler shaders perform better than complex ones. In particular, if
  you can remove an \texttt{if} statement from a shader, that will make
  it run faster. Division and math functions like \texttt{log()} should
  be considered expensive too.
\item
  Always have vertex attrib 0 array enabled. If you draw with vertex
  attrib 0 array disabled, you will force the browser to do complicated
  emulation when running on desktop OpenGL (e.g.~on Mac OS X). This is
  because in desktop OpenGL, nothing gets drawn if vertex attrib 0 is
  not array-enabled. You can use \texttt{bindAttribLocation()} to force
  a vertex attribute to use location 0, and use
  \texttt{enableVertexAttribArray()} to make it array-enabled.
\end{itemize}

​
